\documentclass[oneside, english, 10pt, a4paper]{memoir}
\usepackage{CV_template}

\SetFirstName{Pierre}
\SetLastName{Kibleur}
\SetTitle{Ph.D., Engineer in CSE}
\SetPhone{+33 (0)6 09 90 18 77}
\SetEmail{pierre.kibleur@gmail.com}
\SetNationality{French nationality}
\SetExtra{Driving license B}
\SetAddress{Coupure Links 393, 9000 Ghent, Belgium}

\begin{document}
    \header

\vspace{10pt}
	\emph{With hands-on experience on complex materials and robotics, my ambition is to actively \\ participate in the deployment of automation and technical solutions in the naval industry}
\vspace{15pt}
    
    \section{Experience}
   		\CVentry[UGent Center for X-ray Tomography (UGCT), Ghent, Belgium]
			{\BulletPoint 3D Data analyst \\ Consulting on industrial R\&D, using non-destructive testing to assess quality, product development, and processes. Communication with customers and project management. Group promotion at several events\medskip}
			{2022--present}
			{\BulletPoint Researcher \\ Dynamic testing of fiber-based composite materials with quantitative image processing. Presenter at 6 international conferences; presentation award at ICTMS2022. Gave training on robotics and deep learning}
			{2018--2022}
    
        \CVentry[Confinis AG, Geneva, Switzerland]
			{\BulletPoint Consultant (4 months internship)}{2018--2018}{Writing regulatory compliance of joint prostheses in preparation of marketing application dossiers}{}

        \CVentry[University of Fribourg, Fribourg, Switzerland]
            {\BulletPoint Scientific support staff}{2018--2018}{Robotic modeling and decoding of the primate arm to parameterize a therapeutic brain-computer interface}{}

        \CVentry[G-therapeutics, Lausanne, Switzerland]
            {\BulletPoint Roboticist (9 months internship)}
            {2017--2017}{Programming the Rysen medical robot for gait rehabilitation: defined the C++ control architecture, implemented training tasks for patients, and integrated embedded IMU sensors for real-time state estimation}{}
            
        \CVentry[\'Ecole Polytechnique F\'ed\'erale de Lausanne (EPFL), Lausanne, Switzerland]
            {\BulletPoint Teaching assistant}
            {2015--2016}{Providing support in mathematics for a group of 20 second-year physicists}{}
            
    \section{Education}
 		\CVentry[Ghent University, Ghent, Belgium]
	    {\BulletPoint Ph.D. Bio-science Engineering; thesis on ``4D X-ray micro-tomography investigation of water-induced swelling of wood fiberboards''}
	    {2018--2022}{}{}    
	    
		\CVentry[EPFL, Lausanne, Switzerland]
		{\BulletPoint M.Sc. Computational Science and Engineering; thesis on ``Bio-mechanical model of the primates' upper limb: design of stimulation protocols for the recovery of reaching movements in tetraplegia''\medskip}{2015--2018}
		{\BulletPoint B.Sc. Physics; Erasmus+ exchange at ULB, Brussels}{2011--2015}
        
    \section{Competencies}
        \OneLiner{Coding}{C/C++, Python, Matlab, Bash, shell, CUDA, Basic, C\#, LaTeX}{}
        \OneLiner{Libraries}{Pandas, Scipy, scikit-image, OpenCV, TwinCAT, Keras, PyTorch, TensorFlow, numpy}{}
        %\OneLiner[Imaging]{X-ray CT, dual-energy CT, chemical doping, SEM, SEM-EDX, macro photography}{}
        \OneLiner{Software}{Git, Dragonfly, Avizo, VGStudio Max, Fiji, Abaqus, Solid Works, Fusion 360, Visio}{}
        \OneLiner{Environments}{Linux/Windows, Vim, Atom, Visual Studio, Jupyter, Overleaf, Microsoft Office Suite}{}
        \OneLiner{Soft skills}{Project management, Multidisciplinary collaboration, Creativity, Problem solving}{}
        \OneLiner{Communication}{Author of 24 peer-reviewed articles, regular presenter at conferences and meetings}{}
        
	\section{\TwoColumns[Languages]{Hobbies}}	
		\OneLiner[true]{English/French}{Fluent}{Competition rowing: twice Belgian champion}
		\OneLiner[true]{Russian/Dutch}{Limited proficiency \hfill}{Sailing, flute and saxophone}	        
				
\end{document}


  
  
  
  
  
  
  
  
  
  
  
     
    
    \section{Selected publications (3/24)}
%		\OneLiner[Kibleur, et al.] {``Deep learning segmentation of wood fiber bundles in fiberboards" in Composites Science and Technology}{2022}
%		\OneLiner[Kibleur, et al.] {``Detecting thin adhesive coatings in wood fiber materials with laboratory-based Dual-Energy Computed Tomography (DECT)" in Scientific Reports}{2022}
%		\OneLiner[Sinchuk, Kibleur, et al.] {``Geometrical and deep learning approaches for instance segmentation of CFRP fiber bundles in textile composites" in Composite Structures}{2021}
%		\OneLiner[Rahbar, Kibleur, et al.] {``A 3-D bio-inspired odor source localization and its validation in realistic environmental conditions" in IEEE International Conference on Intelligent Robots and Systems}{2017}
\OneLiner{Composites Science and Technology} {Implemented automatic measurements of 3D aggregates in complex composite fiber materials, doubling the accuracy of other works}{2022}
\OneLiner{Scientific Reports} {Exploited the advanced physico-chemical properties of materials and X-rays to optimize the detectability of adhesives embedded in renewable composite materials}{2022}
\OneLiner{International Conference on Intelligent Robots and Systems} {Developed a 3D algorithm for the autonomous control and navigation of air/sea-borne robots, which has been cited over 30 times}{2017}