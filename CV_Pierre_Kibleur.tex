\documentclass[oneside, english, 10pt, a4paper]{memoir}
\usepackage{CV_template}

\SetFirstName{Pierre}
\SetLastName{Kibleur}
\SetTitle{Ph.D., Engineer in CSE}
\SetPhone{+33 (0)6 09 90 18 77}
\SetEmail{pierre.kibleur@gmail.com}
\SetNationality{French nationality}
\SetExtra{Driving license B}
\SetAddress{Coupure Links 393, 9000 Ghent, Belgium}

\begin{document}
    \header

\vspace{10pt}
	\emph{With hands-on experience on complex materials and robotics, my ambition is to actively \\ participate in the deployment of automation and technical solutions in the naval industry}
\vspace{15pt}
    
    \section{Experience}
   		\CVentry[UGent Center for X-ray Tomography (UGCT), Ghent, Belgium]
			{\BulletPoint 3D Data analyst \\ Consulting on industrial R\&D, using non-destructive testing to assess quality, product development, and processes. Communication with customers and project management. Group promotion at several events\medskip}
			{2022--present}
			{\BulletPoint Researcher \\ Dynamic testing of fiber-based composite materials with quantitative image processing. Presenter at 6 international conferences; presentation award at ICTMS2022. Gave training on robotics and deep learning}
			{2018--2022}
    
        \CVentry[Confinis AG, Geneva, Switzerland]
			{\BulletPoint Consultant (4 months internship)}{2018--2018}{Writing regulatory compliance of joint prostheses in preparation of marketing application dossiers}{}

        \CVentry[University of Fribourg, Fribourg, Switzerland]
            {\BulletPoint Scientific support staff}{2018--2018}{Robotic modeling and decoding of the primate arm to parameterize a therapeutic brain-computer interface}{}

        \CVentry[G-therapeutics, Lausanne, Switzerland]
            {\BulletPoint Roboticist (9 months internship)}
            {2017--2017}{Programming the Rysen medical robot for gait rehabilitation: defined the C++ control architecture, implemented training tasks for patients, and integrated embedded IMU sensors for real-time state estimation}{}
            
        \CVentry[\'Ecole Polytechnique F\'ed\'erale de Lausanne (EPFL), Lausanne, Switzerland]
            {\BulletPoint Teaching assistant}
            {2015--2016}{Providing support in mathematics for a group of 20 second-year physicists}{}
            
    \section{Education}
 		\CVentry[Ghent University, Ghent, Belgium]
	    {\BulletPoint Ph.D. Bio-science Engineering; thesis on ``4D X-ray micro-tomography investigation of water-induced swelling of wood fiberboards''}
	    {2018--2022}{}{}    
	    
		\CVentry[EPFL, Lausanne, Switzerland]
		{\BulletPoint M.Sc. Computational Science and Engineering; thesis on ``Bio-mechanical model of the primates' upper limb: design of stimulation protocols for the recovery of reaching movements in tetraplegia''\medskip}{2015--2018}
		{\BulletPoint B.Sc. Physics; Erasmus+ exchange at ULB, Brussels}{2011--2015}
        
    \section{Competencies}
        \OneLiner{Coding}{C/C++, Python, Matlab, Bash, shell, CUDA, Basic, C\#, LaTeX}{}
        \OneLiner{Libraries}{Pandas, Scipy, scikit-image, OpenCV, TwinCAT, Keras, PyTorch, TensorFlow, numpy}{}
        %\OneLiner[Imaging]{X-ray CT, dual-energy CT, chemical doping, SEM, SEM-EDX, macro photography}{}
        \OneLiner{Software}{Git, Dragonfly, Avizo, VGStudio Max, Fiji, Abaqus, Solid Works, Fusion 360, Visio}{}
        \OneLiner{Environments}{Linux/Windows, Vim, Atom, Visual Studio, Jupyter, Overleaf, Microsoft Office Suite}{}
        \OneLiner{Soft skills}{Project management, Multidisciplinary collaboration, Creativity, Problem solving}{}
        \OneLiner{Communication}{Author of 24 peer-reviewed articles, regular presenter at conferences and meetings}{}
        
	\section{\TwoColumns[Languages]{Hobbies}}	
		\OneLiner[true]{English/French}{Fluent}{Competition rowing: twice Belgian champion}
		\OneLiner[true]{Russian/Dutch}{Limited proficiency \hfill}{Sailing, flute and saxophone}	        
				
\end{document}


  
  
  
  
  
  
  
  
  
  
  
     
    
    \section{Selected publications (3/24)}
%		\OneLiner[Kibleur, et al.] {``Deep learning segmentation of wood fiber bundles in fiberboards" in Composites Science and Technology}{2022}
%		\OneLiner[Kibleur, et al.] {``Detecting thin adhesive coatings in wood fiber materials with laboratory-based Dual-Energy Computed Tomography (DECT)" in Scientific Reports}{2022}
%		\OneLiner[Sinchuk, Kibleur, et al.] {``Geometrical and deep learning approaches for instance segmentation of CFRP fiber bundles in textile composites" in Composite Structures}{2021}
%		\OneLiner[Rahbar, Kibleur, et al.] {``A 3-D bio-inspired odor source localization and its validation in realistic environmental conditions" in IEEE International Conference on Intelligent Robots and Systems}{2017}
\OneLiner{Composites Science and Technology} {Implemented automatic measurements of 3D aggregates in complex composite fiber materials, doubling the accuracy of other works}{2022}
\OneLiner{Scientific Reports} {Exploited the advanced physico-chemical properties of materials and X-rays to optimize the detectability of adhesives embedded in renewable composite materials}{2022}
\OneLiner{International Conference on Intelligent Robots and Systems} {Developed a 3D algorithm for the autonomous control and navigation of air/sea-borne robots, which has been cited over 30 times}{2017}


\section{Peer-reviewed publications}
\begin{itemize}    	    	

	\item Sophie Debaenst, Tamara Jarayseh, Hanna De Saffel, Jan Willem Bek, Matthieu Boone, Ivan Josipovic, \textbf{Pierre Kibleur}, Ronald Y. Kwon, Paul Coucke, and Andy Willaert. Crispant analysis in zebrafish as a tool for rapid functional screening of disease-causing genes for bone fragility, \emph{eLife}, 2024.
	
	\item Caori Organista, Ruizhi Tang, Zhitian Shi, Konstantins Jefimovs, Daniel Josell, Lucia Romano, Simon Spindler, \textbf{Pierre Kibleur}, Benjamin Blykers, Marco Stampanoni, and Matthieu N. Boone. Implementation of a dual-phase grating interferometer for multi-scale characterization of building materials by tunable dark-field imaging, \emph{Scientific Reports}, 2024.

	\item Sofie Dierickx, Siska Genbrugge, Hans Beeckman, Wannes Hubau, \textbf{Pierre Kibleur}, and Jan Van den Bulcke. Non-destructive wood identification using X-ray µCT scanning: which resolution do we need?, \emph{Plant Methods}, 2024.

	\item Zaira Manigrasso, Wannes Goethals, \textbf{Pierre Kibleur}, Matthieu N. Boone, Wilfried Philips, and Jan Aelterman. Image-Based Crack Detection Using Total Variation Strain DVC Regularization, \emph{Applied Sciences}, 2023.

	\item \textbf{Pierre Kibleur}, Benjamin Blykers, Matthieu N. Boone, Luc Van Hoorebeke, Joris Van Acker, and Jan Van den Bulcke. Detecting thin adhesive coatings in wood fiber materials with laboratory-based Dual-Energy Computed Tomography (DECT), \emph{Scientific Reports}, 2022.
	
	\item  {\textbf{Pierre Kibleur}, Zaira Manigrasso, Wannes Goethals, Jan Aelterman, Matthieu N. Boone, Joris Van Acker, and Jan Van den Bulcke}. {Microscopic deformations in MDF swelling: a unique 4D-CT characterization}, \emph{Materials and Structures}, 2022.
	
	\item  {\textbf{Pierre Kibleur}, Jan Aelterman, Matthieu N. Boone, Jan Van den Bulcke, and Joris Van Acker}. {Deep learning segmentation of wood fiber bundles in fiberboards}, \emph{Composites Science and Technology},  2022.
	
	\item Haichao Li, Jan Van den Bulcke, \textbf{Pierre Kibleur}, Orly Mendoza, Stefaan De Neve, and Steven Sleutel. Soil textural control on moisture distribution at the microscale and effect on organic matter mineralization, \emph{Soil Biology and Chemistry}, 2022.
	
	\item {Wanzhao Li, Zheng Zhang, Changtong Mei, \textbf{Pierre Kibleur}, Joris Van Acker, and Jan Van den Bulcke}. {Understanding the mechanical strength and dynamic structural changes of wood-based products using X-ray computed tomography}, \emph{Wood Material Science {\&} Engineering}, 2022.
	
	\item {Yuriy Sinchuk, \textbf{Pierre Kibleur}, Jan Aelterman, Matthieu N. Boone, and Wim Van Paepegem}. {Geometrical and deep learning approaches for instance segmentation of CFRP fiber bundles in textile composites}, \emph{Composite Structures}, 2021.
	
	\item {Jure \v{Z}}igon, Matja{\v{z}} Pavli{\v{c}}, \textbf{Pierre Kibleur}, Jan Van den Bulcke, Marko Petri{\v{c}}, Joris Van Acker, and Sebastian Dahle. {Treatment of wood with atmospheric plasma discharge: study of the treatment process, dynamic wettability and interactions with a waterborne coating}, \emph{{Holzforschung}}, 2021.
	
	\item {\textbf{Pierre Kibleur}, Shravan R.Tata, Nathan Greiner, Sara Conti, Beatrice Barra, Katie Zhuang, Melanie Kaeser, Auke Ijspeert, and Marco Capogrosso}. {Spatiotemporal maps of proprioceptive inputs to the cervical spinal cord during three-dimensional reaching and grasping}, \emph{IEEE Transactions on Neural Systems and Rehabilitation Engineering}, 2020.
	
	\item {Yuriy Sinchuk, \textbf{Pierre Kibleur}, Jan Aelterman, Matthieu N. Boone, and Wim Van Paepegem}. {Variational and deep learning segmentation of very-low-contrast X-ray computed tomography images of carbon/epoxy woven composites}, \emph{Materials}, 2020.   
	
	\item {Wanzhao Li, Zheng Zhang, Guoqiang Zhou, \textbf{Pierre Kibleur}, Changtong Mei, Jiangtao Shi, Joris Van Acker, and Jan Van den Bulcke}. {The effect of structural changes on the compressive strength of LVL}, \emph{Wood Science and Technology}, 2020.     
	
	\item  {Wanzhao Li, Chaoyi Chen, Jiangtao Shi, Changtong Mei, \textbf{Pierre Kibleur}, Joris Van Acker, and Jan Van den Bulcke}. {Understanding the mechanical performance of OSB in compression tests}, \emph{Construction and Building Materials}, 2020.
	
	\item {Gerrit Ralf Surup, Henrik Kofoed Nielsen, Marius Gro{\ss}arth, R{\"{u}}diger Deike, Jan Van den Bulcke, \textbf{Pierre Kibleur}, Michael M{\"{u}}ller, Mirko Ziegner, Elena Yazhenskikh, Sergey Beloshapkin, James J. Leahy, and Anna Trubetskaya}. {Effect of operating conditions and feedstock composition on the properties of manganese oxide or quartz charcoal pellets for the use in ferroalloy industries}, \emph{Energy}, 2020.
	
	\item {Faezeh Rahbar, Ali Marjovi, \textbf{Pierre Kibleur}, and Alfio Martinoli, A 3-D bio-inspired odor source localization and its validation in realistic environmental conditions, \emph{IEEE/RSJ International Conference on Intelligent Robots and Systems (IROS)}, 2017.}
\end{itemize}