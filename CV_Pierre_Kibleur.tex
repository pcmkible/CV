\documentclass[oneside, english, 10pt, a4paper]{memoir}
\usepackage{CV_template}

\SetFirstName{Pierre}
\SetLastName{Kibleur}
\SetAddress{7 Avenue de Saint Mand\'e - 75012 Paris, France}
\SetPhone{+33 (0)6 09 90 18 77}
\SetEmail{pierre.kibleur@epfl.ch}
\SetNationality{French}
\SetExtra{Age 25 (Feb 17th, 1993)}

\begin{document}
    \header
    \section{Education}
        \CVentry[EPFL, Swiss Federal Institute of Technology, Switzerland]
            {Master in Computational Science and Engineering (CSE)}{2015--2018}
            {Bachelor in Physics}{2011--2015}

        \CVentry[ULB, Brussels University, Belgium]
            {Full year Erasmus+ exchange, Physics}{2014--2015}{}{}

        \CVentry[Lyc\'ee Saint-Michel de Picpus, Paris, France]
            {High school diploma in Sciences, with honors}{2011}{}{}

    \section{Experience}
		\CVentry[UNIFR, University of Fribourg, Switzerland]
            {Extension of my Master Thesis's work for preparing a scientific 
            publication.}
            {Mar-(Jul 2018)}{}{}

        \CVentry[EPFL, Biorobotics Laboratory, Lausanne, Switzerland]
            {Biomechanical model of the primates' upper limb: design of stimulation protocols for the
            recovery of reaching movements in tetraplegia (Master Thesis).}{Sep-Jan 2018}{}{}

        \CVentry[G-Therapeutics, Lausanne, Switzerland]
            {Programming of a 3D robotic body weight support system for gait rehabilitation, 
            integration of IMU sensors, writing and automation of the code's unit testing 
            conform to Medical Software norms.}
            {Feb-Sep 2017}{}{}

    \section{Academic projects}         
        \CVentry[EPFL, Distributed Intelligent Systems and Algorithms Laboratory]
            {3D bio-inspired odor source localization algorithm for airborne plumes. Project presented at
            the International Conference on Intelligent Robots and Systems. Ref: EPFL-CONF-231021}{Sep-Jan 2017}{}{}
            
        \CVentry[EPFL, Interdisciplinary Aerodynamics Group]
            {DSMC-CFD coupled simulation of the Stardust capsule's atmospheric re-entry, analysis of the 
            heat diffusion through the Thermal Protection System.}{Feb-Jul 2016}{}{}
        
    \section{Technical skills}
        \OneLiner[Programming]{C/C++, Matlab, Bash, Python, CUDA, Basic}{}
        \OneLiner[Office]{LaTeX, Pack Office, Visio}{}
        \OneLiner[Libraries]{Pandas, Scipy, tikz, TwinCAT, OpenSim}{}
        \OneLiner[Usual environments]{Linux, Vim, Jupyter, Atom, Visual Studio}{}
        \OneLiner[Version control]{Git, Team Foundation Server}

    \section{Academic involvements}
        \OneLiner[Tutoring]{Analysis III for physicists}{Sep-Dec 2016}
        \OneLiner[Class representative]{CSE section}{2015-2016}
    
    \section{Languages}
        \OneLiner[English]{Advanced (C1)}{}
        \OneLiner[Russian]{Basics (A2)}{}
        \OneLiner[French]{Native speaker}{}

\end{document}
